\documentclass[11pt,a4paper]{article}

\usepackage[spanish]{babel}
\usepackage[utf8]{inputenc}
\usepackage{url}


\title{Práctica 2 - Interfaces Gráficas de Usuaria en android}
\author{Interfaces Persona Máquina}
\date{Curso 16/17}

\renewcommand{\abstractname}{Objetivos}


\begin{document}


\maketitle

\begin{abstract}
  Aplicar los conocimientos adquiridos sobre el desarrollo de
  interfaces gráficas de usuaria creando una aplicación para la
  plataforma android.
\end{abstract}


%%%%%%%%%%%%%%%%%%%%%%%%%%%%%%%%%%%%%%%%%%%%%%%%%%%%%%%%%%%%%%%%%%%%%%%%%%%
\section{Descripción}

La aplicación que desarrollarás será un aleatorizador de listas
genérico que facilite la toma de decisiones en grupos.

Los siguientes apartados describen los \emph{sprints} que debes
realizar según la planificación establecida.


\subsection{Requisitos no funcionales}
\begin{itemize}
\item La implementación se realizará con las herramientas y lenguajes
  definidas por la plataforma android.
\end{itemize}


%%%%%%%%%%%%%%%%%%%%%%%%%%%%%%%%%%%%%%%%%%%%%%%%%%%%%%%%%%%%%%%%%%%%%%%%%%%
\section{Sprint 1}

En este \emph{sprint} debes realizar los siguientes pasos:

\begin{enumerate}
\item Diseña una IU que permita ver una lista de categorías y realizar
  las tareas básicas sobre ella: añadir, borrar, editar. Cada
  categoría viene identificada por su nombre.

  Puedes emplear el formato de tu elección para documentar el diseño.

\item Haz un diseño software de la aplicación para dar soporte a la
  interface que acabas de diseñar.

  El diseño debe ajustarse a la idea básica del MVC: separar la vista
  del modelo.

  Para documentar el diseño debes usar los diagramas UML necesarios.

\item Implementa el diseño.

\item Valida todos los pasos anteriores, en especial el funcionamiento
  de tu implementación. A continuación asígnale al último commit del
  repositorio la etiqueta \texttt{sprint1}.

\item Valida el contenido del repositorio remoto\footnote{HINT:
    después de clonarlo, puedes hacer un reset a la etiqueta
    \texttt{sprint1} (\texttt{git clone --hard sprint1})}.
\end{enumerate}


%%%%%%%%%%%%%%%%%%%%%%%%%%%%%%%%%%%%%%%%%%%%%%%%%%%%%%%%%%%%%%%%%%%%%%%%%%%
\section{Sprint 2}

En este \emph{sprint} debes realizar los siguientes pasos:

\begin{enumerate}
\item Cada categoría tiene a su vez asociada una lista de elementos,
  igualmente caracterizados por un nombre.

  Incrementa el diseño de la IU de manera que al seleccionar un
  elemento de la lista de categorías, permita ver su lista de
  elementos asociada. También deben se posible realizar las tareas
  básicas sobre esta última lista: añadir, borrar, editar.

\item Haz un diseño software de la aplicación para dar soporte a la
  interface que acabas de diseñar.

  El diseño debe ajustarse a la idea básica del MVC: separar la vista
  del modelo.

  Para documentar el diseño debes usar los diagramas UML necesarios.

\item Implementa el diseño.

\item Valida todos los pasos anteriores, en especial el funcionamiento
  de tu implementación. A continuación asígnale al último commit del
  repositorio la etiqueta \texttt{sprint2}.

\item Valida el contenido del repositorio remoto.
\end{enumerate}


%%%%%%%%%%%%%%%%%%%%%%%%%%%%%%%%%%%%%%%%%%%%%%%%%%%%%%%%%%%%%%%%%%%%%%%%%%%
\section{Sprint 3}

En este \emph{sprint} debes realizar los siguientes pasos:

\begin{enumerate}
\item Incrementa el diseño de la IU de forma que, mientras se están
  visualizando los elementos de una categoría, sea posible seleccionar
  uno de manera aleatoria y mostrarlo como único elemento en pantalla.

  Para crear una selección aleatoria, debes considerar las siguientes
  acciones:

  \begin{itemize}
  \item Agitar el dispositivo\footnote{Ten en cuenta que no es posible
      en el emulador.}.
  \item Girar el dispositivo (horizontal $\leftrightarrow$ vertical) dos
    veces en menos de un segundo.
  \item Colocar el dispositivo boca abajo, esperar entre uno y dos
    segundos y levantarlo.
  \end{itemize}

\item Incrementa el diseño software de la aplicación para dar soporte
  a los cambios que estás introduciendo.

\item Implementa el diseño.

  OJO: Ten en cuenta que no todos los dispositivos tienen las mismas
  capacidades. Por lo tanto deberás comprobar en ejecución cual de los
  tres métodos es viable.

\item Valida todos los pasos anteriores, en especial el funcionamiento
  de tu implementación. A continuación asígnale al último commit del
  repositorio la etiqueta \texttt{sprint3}.

\item Valida el contenido del repositorio remoto.
\end{enumerate}



%%%%%%%%%%%%%%%%%%%%%%%%%%%%%%%%%%%%%%%%%%%%%%%%%%%%%%%%%%%%%%%%%%%%%%%%%%%
\section{Sprint 4}


En este \emph{sprint} debes realizar los siguientes pasos:

\begin{enumerate}
\item Incrementa el diseño de la IU con un apartado dedicado a
  aprovechar las pantallas de las tablets.

\item Incrementa el diseño software de la aplicación para dar soporte
  a los cambios que estás introduciendo.

\item Implementa el diseño.

  HINT: Si no lo has hecho ya, seguramente tendrás que sustituir tus
  \texttt{Acitivity}s por \texttt{Fragment}s.

\item Valida todos los pasos anteriores, en especial el funcionamiento
  de tu implementación. A continuación asígnale al último commit del
  repositorio la etiqueta \texttt{sprint4}.

\item Valida el contenido del repositorio remoto.
\end{enumerate}





%%%%%%%%%%%%%%%%%%%%%%%%%%%%%%%%%%%%%%%%%%%%%%%%%%%%%%%%%%%%%%%%%%%%%%%%%%%
\section{Sprint 5}


En este \emph{sprint} debes realizar los siguientes pasos:

\begin{enumerate}
\item Internacionaliza la aplicación.

  Usa los mecanismos provistos por la plataforma para el soporte de
  idiomas.

\item Localiza el idioma de la aplicación a dos idiomas de tu
  preferencia.

\item Documenta y corrige los casos en que la interface no cumple el
  principio ``principle of least astonishment''.

\item Documenta y corrige los casos en que la interface:
  \begin{itemize}
  \item no gestiona los errores,
  \item no proporciona \textit{feedback} cuando es necesario,
  \item no gestiona la concurrencia, i.e. se bloquea,
  \item no gestiona el ciclo de vida de la aplicación,
  \end{itemize}

\item Documenta y corrige los casos en que la interface no cumple las
  \emph{Android User Interface Guidelines}.

\item Valida todos los pasos anteriores, en especial el funcionamiento
  de tu implementación. A continuación asígnale al último commit del
  repositorio la etiqueta \texttt{sprint5}.

\item Valida el contenido del repositorio remoto.
\end{enumerate}



\end{document}
